\documentclass[12pt]{article}
\usepackage[english]{babel}
\usepackage[letterpaper,top=2cm,bottom=2cm,left=3cm,right=3cm,marginparwidth=1.75cm]{geometry}
\usepackage{amsmath}
\usepackage{graphicx}
\DeclareMathOperator{\erf}{erf}
\DeclareMathOperator{\probit}{probit}
\title{STAT-UB 103 Homework 6}
\author{Ishan Pranav}
\date{March 21, 2023}
\renewcommand{\thesubsection}{\thesection.\alph{subsection}}
\renewcommand{\theenumi}{\alph{enumi}}
\begin{document}
\maketitle
\section{A recent survey}
Let it be assumed each instance of a credit-card balance paid in full is completely independent of all other instances. Each person either pays or does not pay their credit card bill in full. Since there are a fixed number of independent binary trials ($n$), each with a known probability of success ($p$), the probability that the number of bills paid ($X$) is equal to a given value ($k$) can be modeled with a binomial random variable. 
\[n=400.\]
\[p=0.3.\]

\[P(X=k)=p_k={\binom{n}{k}}p^k(1-p)^{n-k}.\]

\begin{enumerate}
\item
\[P(X\geq k)=\sum^{n}_{i=k}{p_i}=\sum^{n}_{i=k}{{\binom{n}{i}}p^i(1-p)^{n-i}}.\]

\[P(X\geq 110)=\sum^{400}_{i=110}{{\binom{400}{i}}(0.3)^i(0.7)^{400-i}}\approx 0.8745\dots\]
\item\[P(X<k)=\sum^{k-1}_{i=0}{\binom{n}{i}p^i(1-p)^{n-i}}.\]

\begin{align}
P(125\leq X<140)
&=P(X<140)-P(X<125)\\
&=P(X<140)-P(X<125)\\
&=\left[\sum^{139}_{i=0}{\binom{400}{i}(0.3)^i(0.7)^{400-i}}\right]-\left[\sum^{124}_{i=0}{\binom{400}{i}(0.3)^i(0.7)^{400-i}}\right]\\
&\approx 0.2921\dots
\end{align}
\end{enumerate}
\end{document}