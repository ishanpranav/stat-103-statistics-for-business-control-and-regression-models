\documentclass[12pt]{article}
\usepackage[english]{babel}
\usepackage[letterpaper,top=2cm,bottom=2cm,left=3cm,right=3cm,marginparwidth=1.75cm]{geometry}
\usepackage{amsmath}
\usepackage{graphicx}
\DeclareMathOperator{\erf}{erf}
\DeclareMathOperator{\probit}{probit}
\title{STAT-UB 103 Homework 6}
\author{Ishan Pranav}
\date{March 21, 2023}
\renewcommand{\thesubsection}{\thesection.\alph{subsection}}
\renewcommand{\theenumi}{\alph{enumi}}
\begin{document}
\maketitle
\section{A recent survey}
Let it be assumed each instance of a credit-card balance paid in full is completely independent of all other instances. Each person either pays or does not pay their credit card bill in full. Since there are a fixed number of independent binary trials ($n$), each with a known probability of success ($p$), the probability that the number of bills paid ($X$) is equal to a given value ($k$) can be modeled with a binomial random variable. 
\[n=400.\]
\[p\approx 0.30.\]

\[P(X=k)=p_k={\binom{n}{k}}p^k(1-p)^{n-k}.\]

\begin{enumerate}
\item
\[P(X\geq k)=\sum^{n}_{i=k}{p_i}=\sum^{n}_{i=k}{{\binom{n}{i}}p^i(1-p)^{n-i}}.\]

\[P(X\geq 110)\approx\sum^{400}_{i=110}{{\binom{400}{i}}(0.30)^i(0.70)^{400-i}}\approx 0.87.\]
\item\[P(X<k)=\sum^{k-1}_{i=0}{\binom{n}{i}p^i(1-p)^{n-i}}.\]

\begin{align}
P(125\leq X<140)
&=P(X<140)-P(X<125)\\
&=P(X<140)-P(X<125)\\
&\approx\left[\sum^{139}_{i=0}{\binom{400}{i}(0.30)^i(0.70)^{400-i}}\right]-\left[\sum^{124}_{i=0}{\binom{400}{i}(0.30)^i(0.70)^{400-i}}\right]\\
&\approx 0.29.
\end{align}
\end{enumerate}
\section{The daily returns on a portfolio}
Let $X$ represent the daily returns on a portfolio, $\mu_X$ represent the mean daily return, and $\sigma_X$ represent the standard deviation of daily returns.

\[\mu_X\approx 0.001.\]

\[\sigma_X\approx 0.002.\]
\begin{enumerate}
\item The probability that the number of positive-return days of the next 100 days ($Y$) greater than or to a given value ($k$) can be modeled with a binomial random variable.

\[P(Y\geq k)=\sum^{n}_{i=k}{p_i}=\sum^{n}_{i=k}{{\binom{n}{i}}p^i(1-p)^{n-i}}.\]

\[n=100.\]

\[p\approx P(X>0)\approx 1-P(X\leq 0).\]
The probability of a positive portfolio return on a given day ($p$) is dependent on the return itself, which can be modeled with a Gaussian random variable.

\[z_x=\frac{x-\mu_X}{\sigma_X}.\]

\[z_0\approx\frac{0-0.001}{0.002}\approx -0.5.\]

\[\varphi(z)=\frac{e^{-\frac{z^2}{2}}}{\sqrt{2\pi}}.\]

\[\Phi(x)=\int^x_{-\infty}{\varphi(t)\,\mathrm{d}t}=\frac{1}{\sqrt{2\pi}}\int^x_{-\infty}{e^{-\frac{t^2}{2}}\,\mathrm{d}t}.\]

\[P(z\leq x)=\Phi(x).\]

\[p\approx 1-P(z\leq -0.5)\approx 1-\Phi(-0.5)\approx 0.7.\]

\[P(Y\geq 60)\approx\sum^{100}_{i=60}{{\binom{100}{i}}(0.7)^i(0.3)^{100-i}}\approx1.\]
\item The probability that the average return for the portfolio over the next 100 days ($\bar{X}$) exceeds 0.0015 follows from the Central Limit Theorem.

\[\mu_{\bar{X}}=\mu_X\approx 0.001.\]

\[\sigma_{\bar{X}}=\frac{\sigma_X}{\sqrt{n}}\approx\frac{0.002}{\sqrt{100}}\approx 0.0002.\]

\[z_{\bar{x}}=\frac{\bar{x}-\mu_{\bar{X}}}{\sigma_{\bar{X}}}.\]

\[z_{0.0015}\approx\frac{0.0015-0.001}{0.002}\approx 0.25.\]

\begin{align}
P(\bar{X}>0.0015)
&=1-P(\bar{X}\leq 0.0015)\\
&\approx 1-P(z\leq 0.25)\\
&\approx 1-\Phi(0.25)\\
&\approx 0.4.
\end{align}

\end{enumerate}
\end{document}