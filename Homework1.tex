\documentclass[12pt]{article}
\usepackage[english]{babel}
\usepackage[letterpaper,top=2cm,bottom=2cm,left=3cm,right=3cm,marginparwidth=1.75cm]{geometry}
\usepackage{amsmath}
\usepackage{graphicx}
\title{STAT-UB 103 Homework 1}
\author{Ishan Pranav}
\date{January 25, 2023}
\begin{document}
\maketitle
\section{A group of college students}
In this informal study, the treatment variable is the consumption of herbal tea. The students are the experimenters and are treating the residents of the nursing homes, their subjects, with herbal tea. The response variable is the change in residents’ well-being (specifically, their cheerfulness and apparent good health). This is the variable of prime interest in the study. However, the study also offers opportunities for the residents to meet young, bright college students who are interested in their well-being. Social interaction can influence the cheerfulness and health of the residents, so this is an example of a confounding variable. The experiment is fundamentally flawed because the students did not establish a controlled environment with randomized treatment groups. 
\section{An article entitled ``The Science of Polling'' in \emph{Newsweek}, September 28, 1992}
The pollsters encountered nonresponse bias and coverage error. If indeed Republicans were less likely to answer the pollster’s phone call on Friday nights, then there are likely other undiscovered instances of nonresponse bias plaguing the results. The samples taken are not representative of the population. Furthermore, phone polling equates the voting population with the population of landline owners. Anyone who does not own a telephone would have a zero probability of being surveyed. While this may not have been significant in 1992, it would certainly bias the results today.
\section{During World War II}
The engineers' sampling design was susceptible to survivorship bias, which one can interpret as a coverage or sampling error. By selecting ``a large sample of battle-proven airplanes'' or ``a sample of returning planes,'' there is a zero probability that a fallen aircraft would be included in the sample. The sample is not at all representative of the population.

Abraham Wald’s ``odd'' conclusion redefines the population as the set of all planes assumes that survived, making the sample representative of the population. If many returning aircraft had holes in a specific place, then aircraft tend to survive holes there. Therefore, Wald advises reinforcing areas with uncommon holes since the data don't indicate that aircraft tend to survive holes in those areas. 
\end{document}
